\section{注意事项}
子类型的定型可能是编程语言中被广泛误解的概念之最。
子类型的定型原则上来说应该要提供一些方便,让一些程序更容易编写,就像是类型推断那样。
但是包容原则是一把双刃剑。
当$tau'$是$\tau$的子类型时,包容原则允许了$\tau'$到$\tau$的隐式转化。
%implicit passage,不知道应该如何翻译,从上下文来看是某种转化或是过渡
但是当$e$拥有类型$\tau$含有的一些类型时,包容原则削弱了类型断言$e:\tau$的含义。
包容原则妨碍了一些条件的表达,那就是$e$\textit{正是}类型$\tau$,或者两个联结的表达式具有相同的类型。
正是这个弱点造成了子类型的定型的很多麻烦。

对于子类型的定型,有很多已有的文字资料,它们常常和面向对象编程相关。Stamdard ML(Milner et al.,1997)是最早的利用了
子类型系统的语言之一,它的子类型有\texttt{enrichment}和\texttt{realization}两种形式。
%这里的enrichment和realization没有找到现有的翻译。考虑到是专有名词,不敢随意翻译。
前者对应的是积类型的子类型定型,后者对应与类型定义相关的“健忘的”子类型定型(参见章节43)。
%章节43为别的章节的引用
对于子类型的定型的最早的系统性的研究包括Mitchell(1984),Reynolds(1980),和Cardelli(1988)的作品。
Pierce(2002)对子类型的定型给了一个深入的解释,尤其是递归类型和多态类型,并且证明了
有限不可断言的全程量化的子类型的定型是不可判定的。
%bounded impredicative universal quantification直接逐词翻译了
\ctexset{paragraph/runin=false}
\paragraph{习题}
\begin{enumerate}
    \item 对于类型变体
    $$
        (unit \rightharpoonup \tau)\times(\tau\rightharpoonup\texttt(unit))
    $$
    假如将它视为以$\tau$为参数的构造体,它是协变的还是逆变的?对于两者都要给出准确的证明或者是反例。
    \item 考虑两个递归类型
    $$
        \rho_1\triangleq \texttt{rec}\;t\;\texttt{is}\langle\texttt{eq}\hookrightarrow(t\rightharpoonup\texttt{bool})\rangle
    $$
    和
    $$
        \rho_2\triangleq \texttt{rec}\;t\;\texttt{is}\langle\texttt{eq}\hookrightarrow(t\rightharpoonup\texttt{bool}),\texttt{f}\hookrightarrow\texttt{bool}\rangle
    $$
    很明显$\rho_1$不是$\rho_2$的子类型,因为假如将两者视为展开后的积类型,前者的值缺少后者的值所拥有的成分。
    但是,$\rho_2$是否是$\rho_1$的子类型呢?
    如果是,利用小节\ref{section:structural_subtyping_Variance}中的规则给出推导。
    如果不是,给出一个反例说明该子类型的定型会违背类型安全。
    \item 有另一种能够保证安全性,但是赋予了包容原则动态意义的实现子类型的定型的动态特性方法。
    %这句话英文我都没读懂。approach to dynamics是什么意思?是实现动态特性的方法么?
    这种方法采用了一种叫做\textit{强制}的证明方法,对于每个子类型关系,每当包容原则被应用时,插入一个强制子。
    更准确地讲,
    \begin{itemize}
        \item[(a)]对任意一个有效的子类型关系$\tau<:\tau'$,一个强制函数$\chi:\tau\rightharpoonup\tau'$将一个类型为$\tau$的值转化为类型为$\tau'$的值。
        \item[(b)]将包容规则翻译为隐式强制。具体而言,当$\tau<:\tau'$被$\chi:\tau\rightharpoonup\tau'$所证明时,
        对$e:\tau$利用包容原则,并利用函数$\chi$来得到$\chi(e):\tau'$。
    \end{itemize}
    对于“宽”积类型的子类型的定型,以及积类型、和类型和函数类型的变体规则的子类型关系
    准确地改写这个方法。
    你的解答应该明确表述它避免了前文提到的默认投影假设。

    然而,可能有多个强制函数$\chi:\tau\rightharpoonup\tau'$对应子类型关系$\tau<:\tau'$。
    这意味着一个程序可能会在包容原则被运用时,依赖于哪个强制函数被选择来决定其行为。
    如果对每个子类型关系,有且只有一个强制,这种情况叫做\textit{一致}。
    你对积类型的子类型定型的带有强制的翻译是一致的吗?
    (合适地对待一致性需要表达式的等效性,这在章节47中会讨论)
    %此句为字面意思翻译,但是不通顺。
    %47为别的章节的引用
\end{enumerate}
%练习题不知道该怎么编号了