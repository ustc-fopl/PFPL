\section{动态语义}

\textbf{PCF}的动态语义是由\gls{judgment}\(e\ \mathtt{val}\)和\(e \longmapsto e'\)所定义的,前者表示闭值,后者表示求值的一步。

\gls{judgment}\(e\ \mathtt{val}\)是由如下的规则定义的:

\begin{gather}
	\frac{}{
		\mathtt{z}\ \mathtt{val}
	}
	\tag{19.2a} \\
	\frac{
		[e\ \mathtt{val}]
	}{
		\mathtt{s}(e) \mathtt{val}
	}
	\tag{19.2b} 
	\label{equ:bracketed.1} \\
	\frac{}{
		\mathtt{lam}\{\tau\}(x.e) \mathtt{val}
	}
	\tag{19.2c}
\end{gather}

规则\ref{equ:bracketed.1}的带中括号的前提表示它在后继表达式是\gls{eager evaluation}(eager evaluation)的情况中被包含,但在\gls{lazy evaluation}(lazy evaluation)的情况中被忽略。
% 下一句包括章节引用,原文中有链接
(见第三十六章关于惰性的详解)

化简\gls{judgment}(transition judgment)\(e \longmapsto e'\)由以下规则定义:
\begin{gather}
	\bigg[
	\frac{
		e \longmapsto e'
	}{
		\mathtt{s}(e) \longmapsto \mathtt{s}(e')
	}
	\bigg]
	\tag{19.3a}
	\label{equ:bracketed.2} \\
	\frac{
		e \longmapsto e'
	}{
		\mathtt{ifz}\{e_0;x.e_1\}(e) \longmapsto \mathtt{ifz}\{e_0;x.e_1\}(e')
	}
	\tag{19.3b} \\
	\frac{}{
		\mathtt{ifz}\{e_0;x.e_1\}(\mathtt{z}) \longmapsto e_0
	}
	\tag{19.3c} \\
	\frac{
		\mathtt{s}(e)\ \mathtt{val}
	}{
		\mathtt{ifz}\{e_0;x.e_1\}(\mathtt{s}(e)) \longmapsto [e/x]e_1
	}
	\tag{19.3d} \\
	\frac{
		e_1 \longmapsto e_1'
	}{
		\mathtt{ap}(e_1;e_2) \longmapsto \mathtt{ap}(e_1';e_2)
	}
	\tag{19.3e} \\
	\bigg[
	\frac{
		e_1\ \mathtt{val} \quad e_2 \longmapsto e_2'
	}{
		\mathtt{ap}(e_1;e_2) \longmapsto \mathtt{ap}(e_1;e_2')
	}
	\bigg]
	\tag{19.3f} 
	\label{equ:bracketed.3} \\
	\frac{
		[e_2\ \mathtt{val}]
	}{
		\mathtt{ap}(\mathtt{lam}\{\tau\}(x.e);e_2) \longmapsto [e_2/x]e
	}
	\tag{19.3g} 
	\label{equ:bracketed.4} \\
	\frac{}{
		\mathtt{fix}\{\tau\}(x.e) \longmapsto [\mathtt{fix}\{\tau\}(x.e)/x]e
	}
	\tag{19.3h}
	\label{equ:fix.dynamics}
\end{gather}
带中括号的规则\ref{equ:bracketed.2}在后继表达式是\gls{eager evaluation}的情况中被包含,否则应该被忽略。
带中括号的规则\ref{equ:bracketed.3}以及规则\ref{equ:bracketed.4}里带中括号的前提在函数应用的\gls{call-by-value}(call-by-value)时被包含,在\gls{call-by-name}(call-by-name)时被忽略。
规则\ref{equ:fix.dynamics}通过将递归表达式体内的变量\(x\)替换为自身实现了自参照,这个过程称作递归的展开(unwinding)。

\paragraph{引理19.2. 安全性} \hfil

1. 如果\(e: \tau\)并且\(e \longmapsto e'\),那么\(e': \tau\)

2. 如果\(e: \tau\),要么有\(e \mathtt{val}\),要么存在\(e'\)满足\(e \longmapsto e'\)

% 留意一下这里,感觉我按照原文翻译得不太通顺
证明:保留性通过在化简\gls{judgment}推导上的归纳来证明。
考虑规则\ref{equ:fix.dynamic},并假设\(\mathtt{fix}\{\tau\}(x.e): \tau\)。
通过倒转和替换,我们就能得到想要的\([\mathtt{fix}\{\tau\}(x.e)/x]: \tau\)。
同时,通过类型\gls{judgment}推导上的归纳来继续过程上的证明。
例如规则\ref{equ:fix.static}的结果可以延续,因为我们可以通过递归的展开来进行下去。

我们很容易可以发现如果\(e \mathtt{val}\),那么\(e\)是不可规约的因为没有\(e'\)使得\(e \longmapsto e'\)。
安全性定理则反过来说明一个不可规约的表达式是一个值,因为是闭式且\gls{well-typed}。

\gls{call-by-name}型\textbf{PCF}中定义上的等价用\(\Gamma \vdash e_1 \equiv e_2\)来表示,它是最强的全等一致性,包含以下的公理:

\begin{gather}
	\frac{}{
		\Gamma \vdash \mathtt{ifz}\{e_0; x.e_1\}(\mathtt{z}) \equiv e_0: \tau
	}
	\tag{19.4a} \\
	\frac{}{
		\Gamma \vdash \mathtt{ifz}\{e_0; x.e_1\}(\mathtt{s}(e)) \equiv [e/x]e_1: \tau
	}
	\tag{19.4b} \\
	\frac{}{
		\Gamma \vdash \mathtt{fix}\{\tau\}(x.e) \equiv [\mathtt{fix}\{\tau\}(x.e)/x]e: \tau
	}
	\tag{19.4c} \\
	\frac{}{
		\Gamma \vdash \mathtt{ap}(\mathtt{lam}\{\tau\}(x.e_2); e_1) \equiv [e_1/x]e_2: \tau
	}
	\tag{19.4d}
\end{gather}

这些规则满足任何\(\mathtt{nat}\)类型的闭式值的计算:如果\(e: \mathtt{nat}\),那么当且仅当\(e \longmapsto *\bar{n}\)时,有\(e \equiv \bar{n}: \tau\)。