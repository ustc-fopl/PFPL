\section{定义能力}

让我们将递归函数写为\(\mathtt{fun}\ x(y: \tau_1): \tau_2\ \mathtt{is}\ e\),其函数体中\(e:\tau_2\)绑定了两个变量:\(y: \tau_1\)代表参数;\(x: \tau_1 \rightharpoonup \tau_2\)代表函数自身。
该动态语义是由下面的公理定义的
\[
	\frac{}{
		(\mathtt{fun}\ x(y: \tau_1): \tau_2\ \mathtt{is}\ e)(e_1) \longmapsto [\mathtt{fun}\ x(y: \tau_1): \tau_2\ \mathtt{is}\ e, e_1/x, y]e
	}
\]
上式的意思是,应用一个递归函数,我们将\(x\)替换为递归函数自身并将函数体内的\(y\)替换为参数。

递归函数在\textbf{PCF}中使用递归函数来定义,将\(\mathtt{fun}\ x(y: \tau_1): \tau_2\ \mathtt{is}\ e\)写作
\[
	\mathtt{fix}\ x: \tau_1 \rightharpoonup \tau_2\ \mathtt{is}\ \lambda (y: \tau_1)e
\]
我们可以很简单地从这个定义推到出递归函数的动态语义。

\textbf{T}中\gls{primitive recursion}(primitive recursion)的构造可以在\textbf{PCF}中通过递归函数来定义,用表达式
\[
	\mathtt{rec}\ e\ \{\mathtt{z} \hookrightarrow e_0 | \mathtt{s}(x) \mathtt{with} y \hookrightarrow e_1\}
\]
表示函数应用\(e'(e)\),其中\(e'\)就是\gls{general recursion}函数
\[
	\mathtt{fun}\ f(u: \mathtt{nat}): \tau\ \mathtt{is}\ \mathtt{ifz}\ u\ \{z \hookrightarrow e_0 | \mathtt{s}(x) \hookrightarrow [f(x)/y]e_1\}
\]
\gls{primitive recursion}的静态和动态语义可以用这条规则从\textbf{PCF}扩展出来。

一般地,\textbf{PCF}中可定义性的函数是部分的,因为可能会在一些参数上没有定义。
一个数学上的\gls{partial function}\(\phi: \mathbb{N} \rightharpoonup \mathbb{N}\)在\textbf{PCF}上可定义,当且仅当存在一个表达式\(e_\phi : \mathtt{nat} \rightharpoonup \mathtt{nat}\)满足当且仅当\(e_\phi(\overline{m}) \equiv \overline{n}: \mathtt{nat}\)时,有\(\phi(m) = n\)。
举例来说,如果\(\phi\)是一个完全没有定义的函数,那么\(e_\phi\)可以是任何一个在应用时不断循环而永不返回的函数。

区分出那些在\textbf{PCF}中可定义的\gls{partial function}\(\phi\)是十分有益的。
\gls{partial recursive}函数是在\gls{primitive recursion}的基础上扩展了最小化操作:给定\(\phi(m, n)\),定义\(\psi(n)\)为最小的\(m \ge 0\)具有对每个\(m' < m\),\(\phi(m', n)\)有定义且非零,并且\(\phi(m, n) = 0\);
如果没有这样的\(m\),那么\(\psi(n)\)就是未定义的。

\begin{theorem}
	一个自然数域上的\gls{partial function}\(\phi\)在\textbf{PCF}上是可定义的,当且仅当它是\gls{partial recursive}的(partial recursive)。
\end{theorem} 

\begin{proof}
	最小化函数(minimization)是可以在\textbf{PCF}中定义的,所以它至少能定义\gls{partial recursive}函数。
	反过来的证明就有些单调,我们可以通过哥德尔编码来将表达式表示为自然数来判断\textbf{PCF}中定义的函数是\gls{partial recursive}函数。
	这样,\textbf{PCF}的表达能力不会超过\gls{partial recursive}函数的集合。
\end{proof}

% 这里的脚注是带第二十一章的引用链接的
Church定律表明\gls{partial recursive}函数和自然数上的有效计算函数(effectively computable function)集合是一致的,后者是用任一种编程语言写出的或可以写出的程序中存在的计算\footnote{Church定律详见第二十一章}。
因此,\textbf{PCF}在定义自然数域上函数的能力可以和其它任何编程语言媲美。

\textbf{PCF}上的\gls{universal function}(universal function)\(\phi_{univ}\)是一个自然数上的\gls{partial function},定义为
\[
	\phi_{univ}(\ulcorner  e \urcorner)(m) = n \text{,当且仅当} e(\overline{m}) \equiv \overline{n}: \mathtt{nat}
\]
和\textbf{T}相反,\textbf{PCF}中的\gls{universal function}\(\phi_{univ}\)是部分的,而可能在一些输入上没有定义。
本质上,将一个\(\mathtt{nat} \rightharpoonup \mathtt{nat}\)类型闭式的代码\(\overline{\ulcorner e \urcorner}\)给到解释器,就会模拟其动态语义并计算出结果,如果有结果的话,就能应用在\(\overline{m}\)上并获得\(\overline{n}\)。
毕竟这个过程可能并不能终止,\gls{universal function}也不是定义在所有的输入上的。

从Church定律可以看出\gls{universal function}是可以定义在\textbf{PCF}上的。
% 这里有第九章的引用链接
相反,第九章中类似的函数是不能通过对角化的方法定义在\textbf{T}上。
考虑为什么在当前的条件下参数不能被应用是一件有意义的事情。
% 这里有第九章第四节的引用链接
就像在第9.4节中所做的,我们能够推导出这样的等式给\textbf{PCF}
\[
	e_\Delta(\overline{\ulcorner e \urcorner}) \equiv \mathtt{s}(e_\Delta(\overline{\ulcorner e \urcorner}))
\]
但是现在,不但不能想在\textbf{T}上那样得到\gls{universal function}\(e_{univ}\)不存在的结论,我们还能总结出\(e_{univ}\)会在\(e_Delta\)应用在自己的代码上时发散。
