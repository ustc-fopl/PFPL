\section{数据抽象}

为了说明FE的使用,我们考虑一个支持三个操作的自然数队列抽象类型:
\begin{enumerate}

\item 生成空队列。
\item 在队尾插入一个元素。
\item 如果队列非空,删除头结点。

\end{enumerate}

这显然是一个简单的接口,但足以说明数据抽象的主要思想。
%bare-bones翻译成基础更好
虽然队列元素是自然数,但这并没有多大关系。
关键并不是这个队列是什么样子,而是我们可以进行什么操作。
%“可以”后加上“用它”,必须明确是在队列上的操作
队列的行为用存在类型$\exists(t.\tau)$表示为:
$$\exists(t.\left \langle emp \hookrightarrow t,ins \hookrightarrow nat \times t \rightarrow t,
rem \hookrightarrow t \rightarrow (nat \times t)opt \right \rangle)$$
队列的表示类型t是\textit{抽象的}——所有关于它的已知信息就是对于给定的类型它支持emp、ins和rem操作。

队列的实现包括指定了表示类型的包和相关的操作。
在内部实现中,队列的表现形式是已知的并且由操作所依赖。
这是一个非常简单的实现$e_{l}$,其中队列表示为列表:
$$pack\ natlist\ with \left \langle emp \hookrightarrow nil,ins \hookrightarrow e_{i},rem \hookrightarrow e_{r} \right \rangle 
as\ \exists(t.\tau)$$
其中
$$e_{i}:nat \times natlist \rightarrow natlist=\lambda(x:nat \times natlist)...$$
以及
$$e_{r}:natlist \rightarrow nat \times natlist=\lambda(x:natlist)...$$

$e_{i}$的消去体将x的第一个分量,即元素,放到x的第二个分量,即队列中。
而$e_{r}$的消去体则相反,返回的是头元素和队尾的翻转。
这两个操作都“知道”队列被表示为natlist类型的值,并进行相应的编程。

我们也可以给出另一个接口同为$\exists(t.\tau)$的实现$e_{p}$,但是其队列被表示为成对的列表,由队列的“后半部分”与“前半部分”的翻转组成。
这种由两部分组成的表示法避免了每次调用时都需要反转,使操作降低为常数时间:
$$pack\ natlist \times natlist\ with \left \langle emp \hookrightarrow \left \langle nil,nil \right \rangle,
ins \hookrightarrow e_{i},rem \hookrightarrow e_{r} \right \rangle as \exists(t.\tau)$$
在这种情况下,$e_{i}$的类型为
$$nat \times (natlist \times natlist) \rightarrow (natlist \times natlist)$$
$e_{r}$的类型为
$$(natlist \times natlist) \rightarrow nat \times (natlist \times natlist)$$
这两个操作都“知道”队列被表示为$natlist \times natlist$类型的值,并进行相应的编程。

重要的一点是,无论我们选择哪种队列实现,客户端都会进行检查,因为在类型检查过程中,表示类型是隐藏的或者是抽象的。
因此,它不能依赖于它是natlist还是natlist×natlist或其他类型。
也就是说,客户端独立于抽象类型的表示。
