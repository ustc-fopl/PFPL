\section{存在类型的可定义性}

\textbf{FE}语言不是F的适当扩展,因为存在类型(在懒惰动态语义下)可以根据普通类型定义。
%章首universals翻译成了普遍类型,可以统一
为什么这应该成为可能?
请注意,抽象类型的客户端在表示类型中是多态的。
关于
$$open\ e_{1}\ as\ t\ with\ x:\tau\ in\ e_{2}:\tau_{2}$$
其中$e_{1}:\exists(t.\tau)$的类型规则指明,如果$t\ type$并且$x:\tau$,那么$e_{2}:\tau——{2}$。
本质上,客户端是类型的多态函数
$$\forall(t.\tau \rightarrow \tau_{2})$$
其中t可能在$\tau$(操作的类型)中出现,但是不可能在$\tau_{2}$(结果的类型)中出现。
这表明存在类型的以下编码:
$$\exists(t.\tau) \triangleq \forall(u.\forall(t.\tau \rightarrow u) \rightarrow u)$$
$$pack\ \rho\ with\ e\ as\ \exists(t.\tau) \triangleq \Lambda(u)\lambda(x:\forall(t.\tau \rightarrow u))x[\rho](e)$$
$$open\ e_{1}\ as\ t\ with\ x:\tau\ in\ e_{2} \triangleq e_{1}[\tau_{2}](\Lambda(t)\lambda(x:\tau)e_{2})$$
一个存在类型被编码为一个多态函数,它将整体结果类型u作为参数,和一个用结果类型u表示客户端的多态函数,最终产生一个类型为u的值作为总体结果。 因此,open结构简单地将客户端打包成这种多态函数,在结果类型$\tau_{2}$处实例化存在类型,并将其应用于多态客户端。
(因此,编码转换依赖于了解open结构的整体结果类型$\tau_{2}$。)
最后,由表示类型$\rho$和实现$e$组成的包是一个多态函数。
当给定结果类型$u$和客户端$x$时,用$\rho$实例化$x$并传递给它的实现$e$。
%这个句子拆成两半会更通顺
