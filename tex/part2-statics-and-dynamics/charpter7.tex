\chapter{分析动态}


在第5章中,我们用动态结构定义了表达式 $\mathrm{E}$ 的分析。 动态结构对于证明安全性非常有用,
但是出于某些目的,比如编写用户手册,另一个被称为{\it 分析动态}的公式更可取。分析动态是

一个短语与它的值之间的关系,定义时没有详细说明逐步分析的过程。采用成本度量来指定分析的资

源使用情况,用{\it 动态成本}丰富动态分析。一个主要的例子是时间,它是根据表达式的动态结构来计算

表达式所需的转换步骤数。

7.1分析动态

{\it 分析动态}包括分析判断 $e\Downarrow v$ 的归纳定义,其中,封闭表达式 $e$ 的值为 $v$. 用下列规则定义分析动态:
\begin{center}
$\overline{\mathrm{n}\mathrm{u}\mathrm{m}[n]\Downarrow \mathrm{n}\mathrm{u}\mathrm{m}[n]}$   (7.1a)

$\overline{\mathrm{s}\mathrm{t}\mathrm{r}[\mathrm{s}]\Downarrow \mathrm{s}\mathrm{t}\mathrm{r}[\mathrm{s}]}$   (7.1b)

$\displaystyle \frac{e_{1}\Downarrow \mathrm{n}\mathrm{u}\mathrm{m}[n_{1}]e_{2}\Downarrow \mathrm{n}\mathrm{u}\mathrm{m}[n_{2}]n_{1}+n_{2}=n}{\mathrm{p}1\mathrm{u}\mathrm{s}(e_{1};e_{2})\Downarrow \mathrm{n}\mathrm{u}\mathrm{m}[n]}$   (7.1c)

$\displaystyle \frac{e_{1}\Downarrow \mathrm{s}\mathrm{t}\mathrm{r}[\mathrm{s}_{1}]e_{2}\Downarrow \mathrm{s}\mathrm{t}\mathrm{r}[\mathrm{s}_{2}]\mathrm{s}_{1^{\wedge}}\mathrm{s}_{2}=\mathrm{s}}{\mathrm{c}\mathrm{a}\mathrm{t}(e_{1};e_{2})\Downarrow \mathrm{s}\mathrm{t}\mathrm{r}[\mathrm{s}]}$   (7.1d)

$\displaystyle \frac{e\Downarrow \mathrm{s}\mathrm{t}\mathrm{r}[\mathrm{s}]|\mathrm{s}|=n}{1\mathrm{e}\mathrm{n}(e)\Downarrow \mathrm{n}\mathrm{u}\mathrm{m}[n]}$   (7.1e)

$\displaystyle \frac{[e_{1}/x]e_{2}\Downarrow v_{2}}{1\mathrm{e}\mathrm{t}(e_{1};x.e_{2})\Downarrow v_{2}}$   (7.1f)
\end{center}
一个let 表达式的值取决于在绑定在主体中的替代部分。规则不是语法制导的,因为规则(7.1f)的前提不是该规则结论中表达式的子表达式。

规则(7.1f)指定定义的按名解释。对于按值解释,应采用以下规则:

\begin{center}
$\displaystyle \frac{e_{1}\Downarrow v_{1}[v_{1}/x]e_{2}\Downarrow v_{2}}{1\mathrm{e}\mathrm{t}(e_{1};x.e_{2})\Downarrow v_{2}}$   (7.2)
\end{center}
由于分析判断是归纳定义的,我们通过规则引入来证明其性质。具体地说,通过展示 $\mathcal{P}(e\Downarrow v)$ 持有的属性,足以看出 $\mathcal{P}$ 是按规则(7.1)封闭的

1. $\mathcal{P}$(num $[n]\Downarrow \mathrm{n}\mathrm{u}\mathrm{m}[n]$).

2. $\mathcal{P}$ (str $[\mathrm{s}]\Downarrow \mathrm{s}\mathrm{t}\mathrm{r}[\mathrm{s}]$).

3. $\mathcal{P}($plus ($e_{1;}\cdot e_{2})\Downarrow \mathrm{n}\mathrm{u}\mathrm{m}[n])$ , 如果 $\mathcal{P}(e_{1}\Downarrow \mathrm{n}\mathrm{u}\mathrm{m}[n_{1}])$ , $\mathcal{P}(e_{2}\Downarrow \mathrm{n}\mathrm{u}\mathrm{m}[n_{2}])$ , 且 $n_{1}+n_{2}=n.$

4. $\mathcal{P}(\mathrm{c}\mathrm{a}\mathrm{t}(e_{1;}\cdot e_{2})\Downarrow \mathrm{s}\mathrm{t}\mathrm{r}[\mathrm{s}])$ , 如果 $\mathcal{P}(e_{1}\Downarrow \mathrm{s}\mathrm{t}\mathrm{r}[\mathrm{s}1])$ , $\mathcal{P}(e_{2}\Downarrow \mathrm{s}\mathrm{t}\mathrm{r}[\mathrm{s}2])$ , 且 $\mathrm{s}_{1^{\wedge}}\mathrm{s}_{2}=\mathrm{s}.$

5. $\mathcal{P}(\mathrm{l}\mathrm{e}\mathrm{t}(e_{1;}\cdot x.e_{2})\Downarrow v_{2})$ , 如果 $\mathcal{P}([e_{1}/x]e_{2}\Downarrow v_{2})$ .

这一归纳原则与$e$本身的结构归纳不一样,因为分析规则不是语法制导的。

引理 7.1. {\it 如果} $e\Downarrow v$, {\it 那么} $v\iota/\mathrm{a}/.$

{\it 证明}. 通过归纳规则(7.1)。除规则(7.1f)外所有情况是即时的。对于后一种情况,结果
直接遵循以分析规则为前提的归纳假设。 $\square $

7.2 关联结构与分析动态

对于E,我们已经给出了两种不同形式的动态,自然会提出它们是否等价这个问题,但我们
首先要考虑的是我们所谓等价的具体意义。结构动态描述一个逐步执行的过程,而分析动态
则忽略中间状态,只关注初始状态和最终状态。正确的对应关系是结果动态的完整执行序列
对应分析动态中的分析判断。

定理 7.2. {\it 对所有闭式} $e$ {\it 和值} $v, e\mapsto^{*}v$ iff $e\Downarrow v.$

我们应如何证明它?下面将考虑所有方面,首先是最简单的一个

引理 7.3. {\it 如果} $e\Downarrow v$, {\it 那么} $e\mapsto^{*}v.$

{\it 证明}. 通过归纳分析判断的定义。比如,假设plus $(e_{1;}\cdot e_{2}) \Downarrow \mathrm{n}\mathrm{u}\mathrm{m}[n]$ 是分析加法的规则。归纳可知 $e_{1} \mapsto^{*}$

$\mathrm{n}\mathrm{u}\mathrm{m}[n_{1}]$ 和 $e_{2}\mapsto^{*}\mathrm{n}\mathrm{u}\mathrm{m}[n_{2}]$. 理由如下:

plus $(e_{1;}e_{2}) \mapsto^{*}$ plus(num $[n_{1}]_{;}e_{2}$)

$\mapsto^{*}$ plus(num $[n_{1}]_{;}$. num $[n_{2}]$)

$\mapsto$ num $[n_{1}+n_{2}]$

由此有plus $(e_{1;}\cdot e_{2}) \mapsto^{*}\mathrm{n}\mathrm{u}\mathrm{m}[n_{1}+n_{2}]$, 其他情况处理相同. $\square $

对于逆向,回顾第5 章中多步分析和完整分析的定义。因为$v$ val时 $v\Downarrow v$ ,结果表明,在逆向分析下分析是封闭的。

引理 7.4. {\it 如果} $e\mapsto e'$ {\it 且} $e'\Downarrow v$, {\it 那么} $e\Downarrow v.$

{\it 证明}. 通过归纳转化判断的定义。比如,假设在$e_{1} \mapsto$ e\'{i}时 plus $(e_{1;}\cdot e_{2}) \mapsto$plus(e\'{i}; $e_{2}$),假设进一步有plus(e\'{i};$e_{2}$) $\Downarrow v$, 故有 e\'{i} $\Downarrow \mathrm{n}\mathrm{u}\mathrm{m}[n_{1}]$, 且$e_{2}\Downarrow \mathrm{n}\mathrm{u}\mathrm{m}[n_{2}]$, 且 $n_{1}+n_{2}=n$, 且 $v$ 是$\mathrm{n}\mathrm{u}\mathrm{m}[n]$. 通过归纳 $e_{1}\Downarrow \mathrm{n}\mathrm{u}\mathrm{m}[n_{1}]$, 有 plus $(e_{1;}\cdot e_{2})\Downarrow$
$\mathrm{n}\mathrm{u}\mathrm{m}[n]$, 满足要求. $\square $

7.3 类型安全,重查

第6 章给出了类型安全的定义(定理6.1),即保留性与进展性。当将这些概念应用到转换系
统给出的分析动态时,这些概念是有意义的,也正是贯穿全书的做法。但是如果我们在动态
下分析呢?这种情况下如何保证类型安全?

回答是不能。虽然有分析动态的保存属性的类比,但是没有明确的进展属性的类比。保留
性可以这么说,如果 $e \Downarrow v$ 且 $e$ : $\tau$, 那么 $v$ : $\tau$. 通过对分析规则的归纳,可以很容易地证明这一点。但是进展性的类比是什么呢?我们可能会说,如果 $e$ : $\tau$, 那么对于某个 $v$, $e\Downarrow v$ 。虽然这个性质对 $\mathrm{E}$成立,它要求的不仅仅是进展性--它要求每个表达式都有一个值!

如果扩展 $\mathrm{E}$ 以承认可能导致错误的操作(如第6.3 节中所讨论的),或承认不终止的表达式,则此属性无效,即使进展仍然有效。

对于这种情况,一种可能的看法是得出这样的结论:类型安全不能在分析动态的上下文中
得到适当的讨论,而只能通过结构动态来讨论。另一种观点是通过对动态类型错误的显式检
查来检测动态,并且要显示任何带有动态弱类型的表达式都必须是静态弱类型的。在反例中
重新声明,这意味着静态良类型的程序不能导致动态类型错误。这种观点的一个困难是,我
们必须明确地解释一种错误的形式,以证明它不可能出现!然而,可以利用分析动态建立一
种表面的类型安全。

我们定义一个判断 $e$ ?? 表示表达式 $e$在执行时 {\it 出错} 。 ''出错'' 的精确定义通过一组规则给出,但那是在涵盖类型错误对应的所有情况的目的下。以下规则代表一般情况:
\begin{center}
$\overline{\mathrm{p}1\mathrm{u}\mathrm{s}(\mathrm{s}\mathrm{t}\mathrm{r}[\mathrm{s}]_{;}\cdot e_{2})??}$   (7.3a)

$\displaystyle \frac{e_{1}\mathrm{v}\mathrm{a}1}{\mathrm{p}\mathrm{l}\mathrm{u}\mathrm{s}(e_{1};\mathrm{s}\mathrm{t}\mathrm{r}[\mathrm{s}])??}$   (7.3b)
\end{center}
这些规则明确检测附加到字符串中的错误应用;类似的规则控制着语言的每一个原始结
构。

定理 7.5. {\it 如果} $e$ ??, {\it 那么不存在} $\tau$ {\it 满足} $e$ : $\tau.$

{\it 证明}. 通过规则(7.3)的归纳规则得到。例如,规则(7.3a),我们记有 str $[\mathrm{s}]$ : $\mathrm{s}\mathrm{t}\mathrm{r}$,因此 plus(str $[\mathrm{s}]_{;}\cdot e_{2}$) 是弱类型的。  $\square $

推论 7.6. {\it 如果} $e$ : $\tau$, {\it 那么} $\neg (e\ ??)$ .

除了不得不定义判断 $e$ ?? 的不便之外,仅为了表明它对于良类型程序是不可避免的,这
在方法论上有明显的缺陷。如果我们省略了定义判断 $e$ ??的一个或多个规则,定理7.5 的证
明仍然有效;无法确保包含足够多运行时的类型错误检查。我们可以证明我们定义的那些
不能在一个良类型程序中出现,但是我们不能证明我们已经涵盖了所有可能的情况。相比
之下,结构动态不考虑任何弱类型表达式的行为。 因此,任何弱类型表达式都将在没有我
们明确干预的情况下“卡住”,而进展定理排除了所有这种情况。此外,转换系统更接近于
实现--编译器不需要为检测运行时类型错误做任何规定。相反,它依赖静态来确保这些
不会出现,并且不为任何弱类型的程序赋予任何意义。因此,执行效率更高,语言定义也
更简单。

7.4 成本动态

结构动态为程序提供了{\it 时间复杂度}的自然概念,即达到最终状态所需的步骤数。然而,分析
动态并不能提供时间的直接概念。由于不清楚完成分析所需的单个步骤,我们不能直接读出
分析值所需的步骤数。我们必须用成本衡量来扩充分析关系,从而产生{\it 成本动态}。 

分析判断具有形式 $e\Downarrow^{k}v$, 意即 $e$ 通过 $k$步分析出 $v$ 。
\begin{center}
$\mathrm{n}\mathrm{u}\mathrm{m}[n]\Downarrow^{0}\mathrm{n}\mathrm{u}\mathrm{m}[n]$   (7.4a)
\end{center}

\begin{center}
$\displaystyle \frac{e_{1}\Downarrow^{k_{1}}\mathrm{n}\mathrm{u}\mathrm{m}[n_{1}]e_{2}\Downarrow^{k_{2}}\mathrm{n}\mathrm{u}\mathrm{m}[n_{2}]}{\mathrm{p}1\mathrm{u}\mathrm{s}(e_{1};e_{2})\Downarrow^{k_{1}+k_{2}+1}\mathrm{n}\mathrm{u}\mathrm{m}[n_{1}+n_{2}]}$   (7.4b)
\end{center}
str $[\mathrm{s}] \Downarrow^{0}$ str $[\mathrm{s}] (74\mathrm{c})$
\begin{center}
$\displaystyle \frac{e_{1}\Downarrow^{k_{1}}\mathrm{s}_{1}e_{2}\Downarrow^{k_{2}}\mathrm{s}_{2}}{\mathrm{c}\mathrm{a}\mathrm{t}(e_{1};e_{2})\Downarrow^{k_{1}+k_{2}+1}\mathrm{s}\mathrm{t}\mathrm{r}[\mathrm{s}_{1^{\wedge}}\mathrm{s}_{2}]}$   (7.4d)

$\displaystyle \frac{[e_{1}/x].e_{2}\Downarrow^{k_{2}}v_{2}}{1\mathrm{e}\mathrm{t}(e_{1};xe_{2})\Downarrow^{k_{2}+1}v_{2}}$   (7.4e)
\end{center}
对于let 的逐值解释,规则(7.4e)替换为以下规则:
\begin{center}
$\displaystyle \frac{e_{1}\Downarrow^{k_{1}}v_{1}[v_{1}/x]e_{2}\Downarrow^{k_{2}}v_{2}}{1\mathrm{e}\mathrm{t}(e_{1};x.e_{2})\Downarrow^{k_{1}+k_{2}+1}v_{2}}$   (7.5)
\end{center}
定理 7.7. {\it 相同类型的任何闭式} $e$ {\it 和闭值} $v$ , $e\Downarrow^{k}v$ iff $e\mapsto^{k}v.$

{\it 证明}. 从左到右对成本动态的定义进行归纳。从右到左进行自然归纳,通过对结构动态定
义的内部规则归纳。 $\square $

7.5 小结

分析动态和分类规则之间的结构相似性最早出现在{\it 标准} $ML ($ , $)$ {\it 的定义中}。分
析语义的优势在于其直接性;它的缺点是不适合证明类型安全等属性。Robin Milner 引入了恰当的短语“出错”,作为类型错误的描述。Blelloch 和Greiner(1996)在一项并行计算研究中引入了成本动态(见37章)。

习题

7.1. 试说明分析是正确的:  如果 $e\Downarrow v_{1}$ 且 $e\Downarrow v_{2}$, 那么 $v_{1} =v_{2}.$
7.2.  完成引理7.3 的证明 
7.3.  完成引理7.4 的证明.  然后说明:如果 $e\mapsto^{*}e'$ 且 $e'$ val, 那么 $e\Downarrow e'.$

7.4. 在第5 章中,使用 $e$ ?? 表示 $e$ 引起一个检查(或未检查)的错误,用检测错误增强分析动
态。关于类型安全的证明还有何缺陷?你能想出更好的选择吗? 

7.5. 考虑下述形式的一般假设判断
$$
\chi_{1\Downarrow v_{1}},\text{ . }..,\ x_{n}\Downarrow v_{n}\vdash e\Downarrow v
$$
有 $v_{1}$ val, . . ., $v_{n}$ val, 和 $v$ val. 此假设记作 $\Delta$, 被称为环境分析;它们提供了$e$中的自由变量的值。此判断假设$\Delta\vdash e\Downarrow v$ 叫做环境分析动态。
给出一个环境分析动态的归纳定义,不使用任何替代。特别地,应该包含下条来定义自由变
量的取值:
$$
\overline{\Delta,x\Downarrow v\vdash x\Downarrow v}
$$

试证:$x_{1} \Downarrow v_{1}$, . . ., $x_{n} \Downarrow v_{n} \vdash e \Downarrow v$ iff $[v_{1},\ .\ .\ .,\ v_{n}/x_{1},\ .\ .\ .,\ x_{n}]e \Downarrow v$ (使用逐值分析)
