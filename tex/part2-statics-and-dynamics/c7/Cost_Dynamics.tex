\section{7.4 成本化动态语义}%成本化动态语义,下同

结构化动态语义为程序提供了{\it 时间复杂度}的自然概念,即达到最终状态所需的步骤数。然而,求值动态语义并不能提供时间的直接概念。由于不清楚完成分析所需的单个步骤,我们不能直接读出
分析值所需的步骤数。我们必须用成本衡量来扩充分析关系,从而产生{\it 成本化动态语义}。 

求值判断具有形式 $e\Downarrow^{k}v$, 意即 $e$ 通过 $k$步分析出 $v$ 。
\begin{center}
$\mathrm{n}\mathrm{u}\mathrm{m}[n]\Downarrow^{0}\mathrm{n}\mathrm{u}\mathrm{m}[n]$   (7.4a)
\end{center}

\begin{center}
$\displaystyle \frac{e_{1}\Downarrow^{k_{1}}\mathrm{n}\mathrm{u}\mathrm{m}[n_{1}]e_{2}\Downarrow^{k_{2}}\mathrm{n}\mathrm{u}\mathrm{m}[n_{2}]}{\mathrm{p}1\mathrm{u}\mathrm{s}(e_{1};e_{2})\Downarrow^{k_{1}+k_{2}+1}\mathrm{n}\mathrm{u}\mathrm{m}[n_{1}+n_{2}]}$   (7.4b)
\end{center}
str $[\mathrm{s}] \Downarrow^{0}$ str $[\mathrm{s}] (74\mathrm{c})$
\begin{center}
$\displaystyle \frac{e_{1}\Downarrow^{k_{1}}\mathrm{s}_{1}e_{2}\Downarrow^{k_{2}}\mathrm{s}_{2}}{\mathrm{c}\mathrm{a}\mathrm{t}(e_{1};e_{2})\Downarrow^{k_{1}+k_{2}+1}\mathrm{s}\mathrm{t}\mathrm{r}[\mathrm{s}_{1^{\wedge}}\mathrm{s}_{2}]}$   (7.4d)

$\displaystyle \frac{[e_{1}/x].e_{2}\Downarrow^{k_{2}}v_{2}}{1\mathrm{e}\mathrm{t}(e_{1};xe_{2})\Downarrow^{k_{2}+1}v_{2}}$   (7.4e)
\end{center}
对于let 的逐值解释,规则(7.4e)替换为以下规则:
\begin{center}
$\displaystyle \frac{e_{1}\Downarrow^{k_{1}}v_{1}[v_{1}/x]e_{2}\Downarrow^{k_{2}}v_{2}}{1\mathrm{e}\mathrm{t}(e_{1};x.e_{2})\Downarrow^{k_{1}+k_{2}+1}v_{2}}$   (7.5)
\end{center}
定理 7.7. {\it 相同类型的任何闭式} $e$ {\it 和闭值} $v$ , $e\Downarrow^{k}v$ iff $e\mapsto^{k}v.$%iff当且仅当

{\it 证明}. 从左到右对成本动态的定义进行归纳。从右到左进行自然归纳,通过对结构动态定
义的内部规则归纳。 $\square $

